\documentclass[conference]{IEEEtran}
\IEEEoverridecommandlockouts

\usepackage{cite}
\usepackage{amsmath,amssymb,amsfonts}
\usepackage{algorithmic}
\usepackage{graphicx}
\usepackage{textcomp}
\usepackage{xcolor}
\usepackage{hyperref}

\def\BibTeX{{\rm B\kern-.05em{\sc i\kern-.025em b}\kern-.08em
    T\kern-.1667em\lower.7ex\hbox{E}\kern-.125emX}}

\begin{document}

\title{Title}

\author{
\IEEEauthorblockN{Amirmasoud Jafari, Ali Shirazi}
\IEEEauthorblockA{\textit{Hanze University of Applied Sciences}, Groningen, Netherlands \\
Emails: \{a.shirazi, a.jafari\}@st.hanze.nl}
}

\maketitle

\begin{abstract}
Fall detection technology plays a crucial role in enhancing the safety and well-being of individuals, particularly the elderly and those with mobility challenges. Falls are a leading cause of injuries and fatalities among older adults, often resulting in severe health complications and reduced quality of life \cite{worldhealthorganization}. Automatic fall detection systems are designed to promptly identify falls and summon assistance, even when the individual is unable to call for help. These systems utilize advanced sensors, such as accelerometers and gyroscopes, to monitor movements and detect abrupt changes indicative of a fall.

The significance of fall detection lies in its ability to provide immediate response, potentially reducing the severity of injuries and improving outcomes. By automatically alerting emergency services or caregivers, these systems ensure that help is dispatched quickly, which can be critical in preventing long-term health issues and even saving lives. Additionally, fall detection technology offers peace of mind to both users and their families, knowing that assistance is readily available in case of an emergency.

This paper explores the development and implementation of an edge-computed fall detection algorithm for wireless earbuds, leveraging embedded sensors to accurately detect falls and trigger timely interventions. The integration of such technology into everyday devices like earbuds represents a significant advancement in personal safety and healthcare monitoring, making fall detection more accessible and effective for a broader population.
\end{abstract}


% \begin{IEEEkeywords}
% -
% \end{IEEEkeywords}

\section{Introduction}
Falls are a major health risk, particularly for elderly individuals and those with mobility challenges. According to the World Health Organization, falls are the second leading cause of unintentional injury deaths worldwide. Effective fall detection systems can significantly improve safety and quality of life in in-home and assistive care settings. However, existing solutions often rely on expensive or complex setups, making them inaccessible for widespread use \cite{worldhealthorganization}.

Various fall detection methods exist, including computer vision-based systems that use cameras and external sensors. While these approaches can be effective, they raise concerns regarding privacy, cost, and practicality. This research focuses exclusively on the LIS2DW12 accelerometer, a three-axis motion sensor, integrated into Dopple’s EarsOnly Protect2 earbuds. The study explores data collection techniques and machine learning models to identify the most efficient and cost-effective method for detecting falls.

A key challenge in using only an accelerometer is the risk of misclassifying normal movements as falls. To address this, the research investigates ways to enhance user experience and minimize false positives. Additionally, the study considers adaptive modeling, incorporating user behavior patterns and voice detection to improve accuracy and usability.

By leveraging advanced sensors and edge computing, this project aims to develop a reliable, real-time fall detection system that can be seamlessly integrated into everyday devices. This solution has the potential to enhance personal safety and healthcare monitoring on a broader scale while maintaining privacy and accessibility.
\subsection{Research Scope and Main Question}

-----

\section{Literature Review}

-----

\subsection{title}

-----

\subsection{title}

-----

\subsection{title}

----

\subsection{title}

----

\vspace{5pt}
\noindent
----
\vspace{5pt}

\section{Methodology and System Design}

table


\subsection{Research Approach}

-----

\subsection{Real-Time Monitoring and Automated Control Mechanisms}

----

\subsection{Scalability, Security, and Implementation Considerations}

----

\section{Conclusion and Future Work}

----

\bibliographystyle{IEEEtran}
\bibliography{references}

\end{document}